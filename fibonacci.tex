\section{Algorytm macierzowy wyznaczania liczb Fibonacciego}

\comment{W niektórych przypadkach lepiej używać algorytmu dynamicznego (w jakich?)
Algorytm szybkiego potęgowania będzie opisany w tym skrypcie, więc jak już będzie to będziemy chciali się odwołać do odpowiedniego rozdziału a nie do wikipedii.}
W tym rozdziale opiszemy algorytm obliczania liczb Fibonacciego, który wykorzystuje szybkie potęgowanie\footnote{\url{https://en.wikipedia.org/wiki/Exponentiation_by_squaring}}.
Algorytm działa w czasie $O(\log{n})$, co sprawia, że jest znacznie atrakcyjniejszy od algorytmu dynamicznego, który wymaga czasu $O(n)$.

Znajdźmy taką macierz $M$, która po wymnożeniu przez transponowany wektor wyrazów 
$F_{n}$ i $F_{n - 1}$ da nam wektor, w którym otrzymamy wyrazy $F_{n + 1}$ oraz $F_{n}$. 
Łatwo sprawdzić, że dla ciągu Fibonacciego taka macierz ma postać:

\comment{Jeśli nie będziesz korzystać z danego wzoru w rozdziale, to lepiej pominąć jego numerek.
Robi się to przez dodanie symbolu * do equation}
\begin{equation*}
	M = \begin{bmatrix}1 & 1\\1 & 0\end{bmatrix}
\end{equation*}
Bo:
\begin{equation}
\label{eq:fibonacci_m}
	M \times
	\begin{bmatrix}F_n \\ F_{n - 1}\end{bmatrix}
	= \begin{bmatrix}F_{n + 1} \\ F_{n}\end{bmatrix}
\end{equation}

Wynika to wprost z definicji mnożenia macierzy oraz definicji ciągu Fibonacciego.

\comment{Nawiasy okrągłe są zbyt małe. 
Zastąpić je należy $\backslash$left( oraz $\backslash$right).}
\begin{observation}{Zauważmy, że możemy $M$ przemnożyć przez macierz otrzymaną w \ref{eq:fibonacci_m}. Otrzymamy wtedy macierz postaci:}
\begin{equation}
	M \times \left(M \times \begin{bmatrix}F_n \\ F_{n - 1}\end{bmatrix}\right)
\end{equation}
\end{observation}

\comment{To nie są obserwacje.
Obserwacje by były, gdybyś napisał czemu te wzory są równe.}
\begin{observation}{A gdy zrobimy to $n$ razy...}
\label{obs:mult-n-times}
\begin{equation}
	M \times (M \times (M \times ...\, (M \times \begin{bmatrix}F_n \\ F_{n - 1}\end{bmatrix})\,...\,))
\end{equation}
\end{observation}

\begin{fact}{Mnożenie macierzy jest łączne.}
\label{fact:mult-is-associative}
\end{fact}

\comment{Jeśli nie będziesz wstawiał niepotrzebnych enterów to latex nie będzie tak brzydku tabulował tych wierszy.}
Z Faktu \ref{fact:mult-is-associative}. i Obserwacji \ref{obs:mult-n-times}. mamy:

\begin{equation}
	M^n \times \begin{bmatrix}F_n \\ F_{n - 1}\end{bmatrix}
\end{equation}

Pokażemy, że powyższa macierz ma zastosowanie w obliczaniu n-tej liczby Fibonacciego.

\begin{lemma}
\begin{equation}
	M^{n} \times \begin{bmatrix}F_1 \\ F_0\end{bmatrix} = \begin{bmatrix}F_{n + 1} \\ F_{n}\end{bmatrix}
\end{equation}
\end{lemma}

\begin{proof}{Przez indukcję.}\\
Sprawdźmy dla $n = 1$. Mamy:
\begin{equation}
	\begin{bmatrix}1 & 1\\1 & 0\end{bmatrix}^1 \times \begin{bmatrix}1 \\ 0\end{bmatrix}
	= \begin{bmatrix}1 \\ 1\end{bmatrix} = \begin{bmatrix}F_2 \\ F_1\end{bmatrix}
\end{equation}

Rozważmy $n + 1$ zakładając poprawność dla $n$.

\begin{equation}
	\begin{bmatrix}1 & 1\\1 & 0\end{bmatrix}^{n + 1} \times \begin{bmatrix}1 \\ 0\end{bmatrix}
	= \begin{bmatrix}1 & 1\\1 & 0\end{bmatrix} \times \begin{bmatrix}1 & 1\\1 & 0\end{bmatrix}^{n} \times \begin{bmatrix}1 \\ 0\end{bmatrix}
	= \begin{bmatrix}1 & 1\\1 & 0\end{bmatrix} \times \begin{bmatrix}F_{n+1} \\ F_{n}\end{bmatrix}
	\stackrel{\ref{eq:fibonacci_m}}{=} \begin{bmatrix}F_{n + 2} \\ F_{n + 1}\end{bmatrix}
\end{equation}
\end{proof}

\begin{algorithm}[h]
	\DontPrintSemicolon
	\SetAlgorithmName{Algorytm}{}
	
	\KwData{ n }
	
	\KwResult{ $n+1$-sza liczba Fibonacciego }

	$M \leftarrow \begin{bmatrix}1 & 1\\1 & 0\end{bmatrix}$\;
	$M' \leftarrow \texttt{exp\_by\_squaring(M, n)}$\;
	
	\KwRet{$\texttt{pierwszy element wektora}\,(M' \times \begin{bmatrix}1 \\ 0\end{bmatrix})$}\;

	\caption{Procedura \texttt{get\_fibonacci}}
\end{algorithm}

Mimo że powyższy algorytm działa w czasie $O(\log{n})$, warto mieć na uwadze fakt, że liczby Fibonacciego 
rosną wykładniczo. W praktyce oznacza to pracę na liczbach przekraczających długość słowa maszynowego.

\comment{Chcemy do konstrukcji dodać jeszcze wielomiany (aka rozwiązać zadanie z listy?)}
Zaprezentowaną metodę można uogólnić na dowolne ciągi, które zdefiniowane są przez liniową 
kombinację skończonej liczby poprzednich elementów. Wystarczy znaleźć odpowiednią macierz $M$; 
dla ciągów postaci $G_{n + 1} = a_n G_n + a_{n - 1} G_{n - 1} + ... + a_{n - k} G_{n - k}$ jest to:
\begin{equation}
	M = \begin{bmatrix}a_n    & a_{n - 1} & a_{n - 2} & \dots & a_{n - k} & a_{n - k}\\
	                   1      & 0         & 0         & \dots & 0 & 0 \\
	                   0      & 1         & 0         & \dots & 0 & 0\\
	                   0      & 0         & \ddots\\
	                   \vdots &           &           & \ddots\\
	                   0      & 0         & 0         & \dots & 1 & 0
	    \end{bmatrix}
\end{equation}

Dowód tej konstrukcji pozostawiamy czytelnikowi jako ćwiczenie.
