\section{Algorytm macierzowy wyznaczania liczb Fibonacciego}

\comment{Algorytm szybkiego potęgowania będzie opisany w tym skrypcie, więc jak już będzie to będziemy chciali się odwołać do odpowiedniego rozdziału a nie do wikipedii.}
W tym rozdziale opiszemy algorytm obliczania liczb Fibonacciego, który wykorzystuje szybkie potęgowanie\footnote{\url{https://en.wikipedia.org/wiki/Exponentiation_by_squaring}}.
Algorytm działa w czasie $O(\log{n})$, co sprawia, że jest znacznie atrakcyjniejszy (gdy pytamy tylko o jedną liczbę) od algorytmu dynamicznego, który wymaga czasu $O(n)$.

Znajdźmy taką macierz $M$, która po wymnożeniu przez transponowany wektor wyrazów $F_{n}$ i $F_{n - 1}$ da nam wektor, w którym otrzymamy wyrazy $F_{n + 1}$ oraz $F_{n}$. 
Łatwo sprawdzić, że dla ciągu Fibonacciego taka macierz ma postać:
\begin{equation*}
  M = \begin{bmatrix}1 & 1 \\ 1 & 0\end{bmatrix}
\end{equation*}
bo:
\begin{equation}
  \label{eq:fibonacci-m}
  M
  \times
  \begin{bmatrix}F_n \\ F_{n - 1}\end{bmatrix}
  =
  \begin{bmatrix}1 & 1 \\ 1 & 0\end{bmatrix}
  \times
  \begin{bmatrix}F_n \\ F_{n - 1}\end{bmatrix}
  =
  \begin{bmatrix}F_n + F_{n - 1} \\ F_{n}\end{bmatrix}
  =
  \begin{bmatrix}F_{n + 1} \\ F_{n}\end{bmatrix}
\end{equation}
Wynika to wprost z definicji mnożenia macierzy oraz definicji ciągu Fibonacciego.
Wykonajmy mnożenie z równania \ref{eq:fibonacci-m} $n$ razy:
\begin{equation*}
  \underbrace{M \times \bigg(M \times \bigg(M \times ...\,\bigg(M}_\text{n razy}
  \times
  \begin{bmatrix}F_1 \\ F_0\end{bmatrix}\bigg)\,...\,\bigg)\bigg)
\end{equation*}
Z faktu, że mnożenie macierzy jest łączne oraz powyższego wyrażenia otrzymujemy:
\begin{align*}
  M^n \times \begin{bmatrix}F_1 \\ F_0\end{bmatrix}
\end{align*}
Pokażemy, że powyższa macierz ma zastosowanie w obliczaniu n-tej liczby Fibonacciego.
\begin{lemma}
  \begin{equation*}
    M^{n} \times \begin{bmatrix}F_1 \\ F_0\end{bmatrix} = \begin{bmatrix}F_{n + 1} \\ F_{n}\end{bmatrix}
  \end{equation*}
\end{lemma}

\begin{proof}[Dowód przez indukcję]
  Sprawdźmy dla $n = 0$. Mamy:
  \begin{equation*}
    \begin{bmatrix}1 & 1 \\ 1 & 0\end{bmatrix}^0
    \times
    \begin{bmatrix}1 \\ 0\end{bmatrix}
    =
    I
    \times
    \begin{bmatrix}1 \\ 0\end{bmatrix}
    =
    \begin{bmatrix}1 \\ 0\end{bmatrix}
    =
    \begin{bmatrix}F_1 \\ F_0\end{bmatrix}
  \end{equation*}
  Rozważmy $n + 1$ zakładając poprawność dla $n$.
  \begin{equation*}
    \begin{bmatrix}
      1 & 1 \\
      1 & 0
    \end{bmatrix}^{n + 1}
    \times
    \begin{bmatrix}1 \\ 0\end{bmatrix}
    =
    \begin{bmatrix}
      1 & 1 \\
      1 & 0
    \end{bmatrix}
    \times
    \begin{bmatrix}
      1 & 1 \\
      1 & 0
    \end{bmatrix}^{n}
    \times
    \begin{bmatrix}1 \\ 0\end{bmatrix}
    \stackrel{teza}{=}
    \begin{bmatrix}
      1 & 1 \\
      1 & 0
    \end{bmatrix}
    \times
    \begin{bmatrix}F_{n+1} \\ F_{n}\end{bmatrix}
    \stackrel{\ref{eq:fibonacci-m}}{=}
    \begin{bmatrix}F_{n + 2} \\ F_{n + 1}\end{bmatrix}
  \end{equation*}
\end{proof}

\begin{algorithm}[h]
  \DontPrintSemicolon
  \SetAlgorithmName{Algorytm}{}

  \KwData{ n }

  \KwResult{ $n$-ta liczba Fibonacciego }

  $M \leftarrow \begin{bmatrix}
                  1 & 1 \\
                  1 & 0
                \end{bmatrix}$\;
  $M' \leftarrow \texttt{exp\_by\_squaring(M, n - 1)}$\;
  $M'' \leftarrow M' \times \begin{bmatrix}1 \\ 0\end{bmatrix}$\;
  \KwRet{$M''_{1,1}$}\;

  \caption{Procedura \texttt{get\_fibonacci}}
\end{algorithm}
Mimo że powyższy algorytm działa w czasie $O(\log{n})$, warto mieć na uwadze fakt, że liczby Fibonacciego 
rosną wykładniczo. W praktyce oznacza to pracę na liczbach przekraczających długość słowa maszynowego.

Zaprezentowaną metodę można uogólnić na dowolne ciągi, które zdefiniowane są przez liniową 
kombinację skończonej liczby poprzednich elementów. Wystarczy znaleźć odpowiednią macierz $M$; 
dla ciągów postaci:
\begin{equation*}
  G_{n + 1} = a_n G_n + a_{n - 1} G_{n - 1} + ... + a_{n - k} G_{n - k}
\end{equation*}
wygląda ona następująco:
\begin{align*}
  M
  =
  \begin{bmatrix}
    a_n    & a_{n - 1} & a_{n - 2} & \dots  & a_{n - k + 1} & a_{n - k} \\
    1      & 0         & 0         & \dots  & 0             & 0 \\
    0      & 1         & 0         & \dots  & 0             & 0 \\
    0      & 0         & 1         & \dots  & 0             & 0 \\
    \vdots & \vdots    & \vdots    & \ddots & \vdots        & \vdots \\
    0      & 0         & 0         & \dots  & 1             & 0
  \end{bmatrix}
\end{align*}
Dowód tej konstrukcji pozostawiamy czytelnikowi jako ćwiczenie.
