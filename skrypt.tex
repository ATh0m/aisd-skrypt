\documentclass[10pt,b5paper]{book}
\usepackage{polski}
\usepackage{amsthm}
\usepackage[T1]{fontenc}
\usepackage[utf8]{inputenc}
\usepackage{geometry}
\usepackage{tikz}
\usepackage{amsmath}
\usepackage{array}
\usepackage[titletoc]{appendix}
\usepackage{scrextend}
\usepackage[ruled,vlined,commentsnumbered]{algorithm2e}

%% Użyteczne komendy

\newtheorem{theorem}{Twierdzenie}
\newcommand{\komentarz}[1]{\textcolor{red}{(#1)}}
\def\checkmark{\tikz\fill[scale=0.4](0,.35) -- (.25,0) -- (1,.7) -- (.25,.15) -- cycle;} 

%% Strona tytułowa

\usepackage[load-headings]{exsheets}
\DeclareInstance{exsheets-heading}{mylist}{default}{
  runin = true ,
  attach = {
    main[l,vc]number[l,vc](-3em,0pt) ;
    main[r,vc]points[l,vc](\linewidth+\marginparsep,0pt)
  }
}

\SetupExSheets{
  headings = mylist ,
  headings-format = \normalfont ,
  counter-format = se.qu ,
  counter-within = section
}

\usepackage{etoolbox}
\AtBeginEnvironment{question}{\addmargin[3em]{0em}}
\AtEndEnvironment{question}{\endaddmargin}

\newcommand*{\titleGM}{\begingroup
\hbox{
\hspace*{0.2\textwidth}
\hspace*{0.05\textwidth}
\parbox[b]{0.75\textwidth}{
{\noindent\Huge\bfseries Skrypt \\[0.2\baselineskip] z Algorytmów \\[0.2\baselineskip] i struktur danych}\\[1\baselineskip]
{\large \textit{Zbiór mniej lub bardziej ciekawych algorytmów i struktur danych, jakie bywały omawiane na wykładzie (albo nie).}}\\[3\baselineskip]
{\Large \textsc{praca zbiorowa pod redakcją \\[0.1\baselineskip] Krzysztofa Piecucha}}

\vspace{0.5\textheight}
{\noindent Korzystać na własną odpowiedzialność.}\\[\baselineskip]
}}
\endgroup}

\definecolor{titlepagecolor}{cmyk}{0.9,1,.6,.40}

\newcommand\titlepagedecoration{%
\begin{tikzpicture}[remember picture,overlay,shorten >= -10pt]

\coordinate (aux1) at ([yshift=-15pt]current page.north west);
\coordinate (aux2) at ([yshift=-410pt]current page.north west);
\coordinate (aux3) at ([xshift=+4.5cm]current page.north west);
\coordinate (aux4) at ([yshift=-150pt]current page.north west);

\begin{scope}[titlepagecolor!60,line width=14pt,rounded corners=12pt]
\draw
  (aux1) -- coordinate (a)
  ++(-45:5) --
  ++(225:5.1) coordinate (b);
\draw[shorten <= -10pt]
  (aux3) --
  (a) --
  (aux1);
\draw[opacity=0.4,titlepagecolor,shorten <= -10pt]
  (b) --
  ++(-45:2.2) --
  ++(225:2.2);
\end{scope}
\draw[titlepagecolor,line width=9pt,rounded corners=8pt,shorten <= -10pt]
  (aux4) --
  ++(-45:1.2) --
  ++(225:1.2);
\begin{scope}[titlepagecolor!70,line width=6pt,rounded corners=6pt]
\draw[shorten <= -10pt]
  (aux2) --
  ++(-45:3) coordinate[pos=0.45] (c) --
  ++(225:3.1);
\draw
  (aux2) --
  (c) --
  ++(45:2.5) --
  ++(135:2.5) --
  ++(225:2.5) coordinate[pos=0.3] (d);   
\draw 
  (d) -- +(135:1);
\end{scope}
\end{tikzpicture}%
}

\begin{document}

\pagestyle{empty}
\titleGM
\titlepagedecoration

\chapter{To trzeba będzie mądrze podzielić na rozdziały, na razie nie wiem jak}

\section{Algorytm rosyjskich wieśniaków}



Algorytm rosyjskich wieśniaków jest przypisywany sposobowi mnożenia liczb używanemu w XIX-wiecznej Rosji.
Aktualnie jest on stosowany w niektórych układach mnożących.
Pomnożenie dwóch liczb naturalnych $a$ i $b$ jest proste: pierwszą liczbę kolejno dzielimy całkowiecie przez 2, a drugą podwajamy.
\comment{ Instructions not clear.
Poprzedni opis był lepszy.
Ponadto starajmy się unikać sformułowań, że coś jest ``łatwe''.}
Wynik $a \cdot b$ otrzymujemy poprzez zsumowanie takich pośrednich wartości b, dla których odowiadająca im wartość $a$ jest nieparzysta.
Uzyskany w ten sposób wynik jest równy $a \cdot b$.
W poniższym przykładzie obliczymy $42 \cdot 17$.

\begin{center}
\begin{tabular}{ |c|c|c| } 
 \hline
 Iteracja & a & b \\ 
 \hline
 0 & 42 & 17 \\ 
 1 & 21 & $17 \cdot 2 = 34$ \\
 2 & 10 & $17 \cdot 2^2 = 68$ \\
 3 & 5  & $17 \cdot 2^3 = 136$ \\
 4 & 2  & $17 \cdot 2^4 = 272$ \\
 5 & 1  & $17 \cdot 2^5 = 544$ \\
 
 \hline
\end{tabular}
\end{center}

\comment{Jak usuniecie enter po tabelce, to to zdanie nie dostanie w pdfie tabulatora.
Tabulator powinien się znajdować przed akapitami.
To zdanie jest częścią poprzedniego akapitu.}
$a$ jest nieparzyste w iteracji 1, 3 oraz 5. Zatem będziemy sumować wartości $b$ z iteracji 1, 3 i 5.

\begin{equation*} 
\begin{split}
a \cdot b &= 17\cdot2 + 17\cdot2^3 + 17\cdot2^5 \\
&= 34 + 136 + 544 \\
&= 714
\end{split}
\end{equation*}
Faktycznie, otrzymaliśmy wynik poprawny. Spójrzmy raz jeszcze na na tę sumę:

\begin{equation*} 
\begin{split}
a \cdot b &= 17\cdot2 + 17\cdot2^3 + 17\cdot2^5 \\
&= 17 \cdot ( 2^5 + 2^3 + 2) \\
&= 17 \cdot ( 1 \cdot 2^5 + 0 \cdot 2^4 + 1 \cdot 2^3 + 0 \cdot 2^2 + 1 \cdot 2^1 + 0 \cdot 2^0 ) \\
&=17_{10} \cdot 101010_2 \\
&=17 \cdot 42 = 714
\end{split}
\end{equation*}

Przypomnij sobie algorytm zamiany liczby w systemie dziesiętnym na system binarny.
Okazuje się, że algorytm rosyjskich wieśniaków "po cichu" wylicza tę reprezentację $a$.

W kolejnych paragrafach podamy algorytm i w celu jego udowodnienia sformułujmy \textit{niezmiennik} oraz wykażemy jego prawdziwośc.

\begin{algorithm}[h]
  \DontPrintSemicolon
  \SetAlgorithmName{Algorytm}{}
  
  \KwData{ $a$, $b$ - liczby naturalne }
  
  \KwResult{ $wynik = a \cdot b$ }
  
  $a' \leftarrow a$\;
  $b' \leftarrow b$\;
  $wynik \leftarrow 0$\;
  \While{\upshape $a' > 0$}
  {
    \If{\upshape $a' \textsf{ mod } 2 = 1$}
    {
      $wynik \leftarrow wynik + b'$\;
    }
    $a' \leftarrow a' \textsf{ div } 2$\;
    $b' \leftarrow b' \cdot 2$\;
  }
  
  \caption{Algorytm rosyjskich wieśniaków}
  \label{alg-wiesniakow}
\end{algorithm}

\begin{theorem}
Niech $a'_i$ (kolejno: $b'_i$, $wynik_i$) będzie wartością zmiennej \texttt{a'} (\texttt{b'}, \texttt{wynik}) w $i-tej$ iteracji pętli \texttt{while}. Zachodzi następujący niezmiennik pętli:
\[
a'_i \cdot b'_i + wynik_i = a \cdot b \enspace.
\]
\end{theorem}

\begin{lemma}
Przed wejściem do pętli \texttt{while} niezmiennik jest prawdziwy.
\end{lemma}
\begin{proof}
Skoro przed przed wejściem do pętli mamy: $a'_0 = a$, $b'_0 = b$ oraz $wynik_0 = 0$, to oczywiście: $a'_0 \cdot b'_0 + wynik_0 = a \cdot b + 0 = a \cdot b$.
\end{proof}

\begin{lemma}
Po $i-tym$ obrocie pętli niezmiennik jest spełniony.
\end{lemma}
\begin{proof}
Załóżmy, że niezmiennik zachodzi w $i-tej$ iteracji i sprawdźmy co dzieje się w $i+1$ iteracji.
\comment{Nie możesz czegoś takiego założyć :)}
Rozważmy dwa przypadki.


\begin{itemize}
    \item $a'_i$ parzyste. Instrukcja \texttt{if} się nie wykona, w $i+1$ iteracji $wynik_i$ pozostanie niezmieniony, $a'_i$ zmniejszy się o połowę, a $b'_i$ zwiększy dwukrotnie. 
    \[
      wynik_{i+1} = wynik_i
    \]
    \[
      a'_{i+1} = a'_i \textsf{ div } 2 = \frac{a'_i}{2}
    \]
    \[
      b'_{i+1} = b'_i \cdot 2
    \]
    W tym przypadku otrzymujemy:
    \[
      a'_{i+1} \cdot b'_{i+1} + wynik_{i+1} = \frac{a'_i}{2} \cdot 2 b'_i + wynik_i = a'_i \cdot b'_i + wynik_i = a \cdot b
    \]

    \item $a'_i$ nieparzyste:
    \[
      wynik_{i+1} = wynik_i + b'_i
    \]
    \[
      a'_{i+1} = a'_i \textsf{ div } 2 = \frac{a'_i-1}{2}
    \]
    \[
      b'_{i+1} = b'_i \cdot 2
    \]
    Ostatecznie otrzymujemy:
    \[
      a'_{i+1} \cdot b'_{i+1} + wynik_{i+1} = \frac{a'_i-1}{2} \cdot 2 b'_i + wynik_i +b'_i = a'_i \cdot wynik_i + b'_i= a \cdot b
    \]

\end{itemize}

\end{proof}

\begin{lemma}
Po wyjściu z pętli \texttt{while} niezmiennik również jest spełniony.
\comment{Ten lemat też nie brzmi tak jak powinien.}
\end{lemma}
\begin{proof}
Wystarczy zauważyć, że tuż po wyjściu z pętli \texttt{while} wartość zmiennej $a'$ wynosi $0$.
Podstawiając do niezmiennika okazuje się, że faktycznie algorytm rosyjskich wieśniaków liczy $a \cdot b$.
\end{proof}

\begin{lemma}
Algorytm sie kończy.
\end{lemma}
\begin{proof}
Skoro $a_i \in \mathbb{N} $ oraz $\mathbb{N}$ jest dobrze uporządkowany, to połowiąc $a_i$ po pewnej liczbie iteracji otrzymamy 0.
\end{proof}

\comment{Zdanie podsumowujące?
Z powyższych lematów wynika poprawność algorytmu bla bla bla.}

\paragraph{Złożoność}

Z każdą iteracją połowimy $a'$. 
Biorąc pod uwagę kryterium jednorodne pozostałe instrukcje w pętli nic nie kosztują. 
Stąd złożoność to $O(\log a)$.

W kryterium logarytmicznym musimy uwzględnić czas dominującej instrukcji: dodawania  $wynik \leftarrow wynik + b'$. 
W najgorszym przypadku zajmuje ono $O(\log ab)$. Zatem złożoność to $O(\log a \cdot \log ab)$.



\section{Algorytm macierzowy wyznaczania liczb Fibonacciego}

\comment{W niektórych przypadkach lepiej używać algorytmu dynamicznego (w jakich?)
Algorytm szybkiego potęgowania będzie opisany w tym skrypcie, więc jak już będzie to będziemy chciali się odwołać do odpowiedniego rozdziału a nie do wikipedii.}
W tym rozdziale opiszemy algorytm obliczania liczb Fibonacciego, który wykorzystuje szybkie potęgowanie\footnote{\url{https://en.wikipedia.org/wiki/Exponentiation_by_squaring}}.
Algorytm działa w czasie $O(\log{n})$, co sprawia, że jest znacznie atrakcyjniejszy od algorytmu dynamicznego, który wymaga czasu $O(n)$.

Znajdźmy taką macierz $M$, która po wymnożeniu przez transponowany wektor wyrazów 
$F_{n}$ i $F_{n - 1}$ da nam wektor, w którym otrzymamy wyrazy $F_{n + 1}$ oraz $F_{n}$. 
Łatwo sprawdzić, że dla ciągu Fibonacciego taka macierz ma postać:

\comment{Jeśli nie będziesz korzystać z danego wzoru w rozdziale, to lepiej pominąć jego numerek.
Robi się to przez dodanie symbolu * do equation}
\begin{equation*}
	M = \begin{bmatrix}1 & 1\\1 & 0\end{bmatrix}
\end{equation*}
Bo:
\begin{equation}
\label{eq:fibonacci_m}
	M \times
	\begin{bmatrix}F_n \\ F_{n - 1}\end{bmatrix}
	= \begin{bmatrix}F_{n + 1} \\ F_{n}\end{bmatrix}
\end{equation}

Wynika to wprost z definicji mnożenia macierzy oraz definicji ciągu Fibonacciego.

\comment{Nawiasy okrągłe są zbyt małe.
Zastąpić je należy \backslash left( oraz \backslash right).}
\begin{observation}{Zauważmy, że możemy $M$ przemnożyć przez macierz otrzymaną w \ref{eq:fibonacci_m}. Otrzymamy wtedy macierz postaci:}
\begin{equation}
	M \times \left(M \times \begin{bmatrix}F_n \\ F_{n - 1}\end{bmatrix}\right)
\end{equation}
\end{observation}

\comment{To nie są obserwacje.
Obserwacje by były, gdybyś napisał czemu te wzory są równe.}
\begin{observation}{A gdy zrobimy to $n$ razy...}
\label{obs:mult-n-times}
\begin{equation}
	M \times (M \times (M \times ...\, (M \times \begin{bmatrix}F_n \\ F_{n - 1}\end{bmatrix})\,...\,))
\end{equation}
\end{observation}

\begin{fact}{Mnożenie macierzy jest łączne.}
\label{fact:mult-is-associative}
\end{fact}

\comment{Jeśli nie będziesz wstawiał niepotrzebnych enterów to latex nie będzie tak brzydku tabulował tych wierszy.}
Z Faktu \ref{fact:mult-is-associative}. i Obserwacji \ref{obs:mult-n-times}. mamy:

\begin{equation}
	M^n \times \begin{bmatrix}F_n \\ F_{n - 1}\end{bmatrix}
\end{equation}

Pokażemy, że powyższa macierz ma zastosowanie w obliczaniu n-tej liczby Fibonacciego.

\begin{lemma}
\begin{equation}
	M^{n} \times \begin{bmatrix}F_1 \\ F_0\end{bmatrix} = \begin{bmatrix}F_{n + 1} \\ F_{n}\end{bmatrix}
\end{equation}
\end{lemma}

\begin{proof}{Przez indukcję.}\\
Sprawdźmy dla $n = 1$. Mamy:
\begin{equation}
	\begin{bmatrix}1 & 1\\1 & 0\end{bmatrix}^1 \times \begin{bmatrix}1 \\ 0\end{bmatrix}
	= \begin{bmatrix}1 \\ 1\end{bmatrix} = \begin{bmatrix}F_2 \\ F_1\end{bmatrix}
\end{equation}

Rozważmy $n + 1$ zakładając poprawność dla $n$.

\begin{equation}
	\begin{bmatrix}1 & 1\\1 & 0\end{bmatrix}^{n + 1} \times \begin{bmatrix}1 \\ 0\end{bmatrix}
	= \begin{bmatrix}1 & 1\\1 & 0\end{bmatrix} \times \begin{bmatrix}1 & 1\\1 & 0\end{bmatrix}^{n} \times \begin{bmatrix}1 \\ 0\end{bmatrix}
	= \begin{bmatrix}1 & 1\\1 & 0\end{bmatrix} \times \begin{bmatrix}F_{n+1} \\ F_{n}\end{bmatrix}
	\stackrel{\ref{eq:fibonacci_m}}{=} \begin{bmatrix}F_{n + 2} \\ F_{n + 1}\end{bmatrix}
\end{equation}
\end{proof}

\begin{algorithm}[h]
	\DontPrintSemicolon
	\SetAlgorithmName{Algorytm}{}
	
	\KwData{ n }
	
	\KwResult{ $n+1$-sza liczba Fibonacciego }

	$M \leftarrow \begin{bmatrix}1 & 1\\1 & 0\end{bmatrix}$\;
	$M' \leftarrow \texttt{exp\_by\_squaring(M, n)}$\;
	
	\KwRet{$\texttt{pierwszy element wektora}\,(M' \times \begin{bmatrix}1 \\ 0\end{bmatrix})$}\;

	\caption{Procedura \texttt{get\_fibonacci}}
\end{algorithm}

Mimo że powyższy algorytm działa w czasie $O(\log{n})$, warto mieć na uwadze fakt, że liczby Fibonacciego 
rosną wykładniczo. W praktyce oznacza to pracę na liczbach przekraczających długość słowa maszynowego.

\comment{Chcemy do konstrukcji dodać jeszcze wielomiany (aka rozwiązać zadanie z listy?)}
Zaprezentowaną metodę można uogólnić na dowolne ciągi, które zdefiniowane są przez liniową 
kombinację skończonej liczby poprzednich elementów. Wystarczy znaleźć odpowiednią macierz $M$; 
dla ciągów postaci $G_{n + 1} = a_n G_n + a_{n - 1} G_{n - 1} + ... + a_{n - k} G_{n - k}$ jest to:
\begin{equation}
	M = \begin{bmatrix}a_n    & a_{n - 1} & a_{n - 2} & \dots & a_{n - k} & a_{n - k}\\
	                   1      & 0         & 0         & \dots & 0 & 0 \\
	                   0      & 1         & 0         & \dots & 0 & 0\\
	                   0      & 0         & \ddots\\
	                   \vdots &           &           & \ddots\\
	                   0      & 0         & 0         & \dots & 1 & 0
	    \end{bmatrix}
\end{equation}

Dowód tej konstrukcji pozostawiamy czytelnikowi jako ćwiczenie.


%% Dodatki

\begin{appendices}

\chapter{Porównanie programów przedmiotu AiSD na różnych uczelniach}

\begin{center}
\begin{tabular}{lccccc}
 & UWr & UW & UJ & MIT & Oxford \\
Stosy, kolejki, listy &  & \checkmark &   &   &  \\
Dziel i zwyciężaj & \checkmark &   &  &   &  \\
Programowanie Dynamiczne & \checkmark & \checkmark &  \checkmark &  \checkmark &  \\
Metoda Zachłanna & \checkmark & \checkmark & \checkmark &   &  \\
Koszt zamortyzowany & \checkmark & \checkmark &   &   & \checkmark \\
NP-zupełność & \checkmark & \checkmark &   & \checkmark  &  \\
PRAM / NC & \checkmark &  &   &   &  \\
Sortowanie & \checkmark & \checkmark &   &   &  \\
Selekcja & \checkmark & \checkmark &   &   &  \\
Słowniki & \checkmark & \checkmark & \checkmark  &   & \checkmark \\
Kolejki priorytetowe & \checkmark & \checkmark &   &   &  \\
Hashowanie & \checkmark & \checkmark &   &   &  \\
Zbiory rozłączne & \checkmark &  &   &   &  \\
Algorytmy grafowe & \checkmark & \checkmark & \checkmark  & \checkmark  & \checkmark \\
Algorytmy tekstowe & \checkmark & \checkmark &   &   &  \\
Geometria obliczeniowa & \checkmark &  &   &   &  \\
FFT & \checkmark &  &   &   & \checkmark \\
Algorytm Karatsuby & \checkmark &  &   &  \checkmark &  \\
Metoda Newtona &  &  &   & \checkmark  &  \\
Algorytmy randomizowane & \checkmark &  &   &   & \checkmark \\
Programowanie liniowe &  &  &   &   & \checkmark \\
Algorytmy aproksymacyjne & \checkmark &  &   &   & \checkmark \\
Sieci komparatorów & \checkmark &  &   &   &  \\
Obwody logiczne & \checkmark &  &   &   &  \\
\end{tabular}
\end{center}

\end{appendices}

\end{document}