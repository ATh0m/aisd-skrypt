\section{Kopce binarne}

\comment{Chciałbym, aby skrypt był skryptem i żebyśmy tam pisali formalnie.
Dlatego definicję kopca chciałbym mieć zapisaną formalnie (bez przypisów) i w znacznikach \\begin\{definition\} \\end\{definition\}.
Teraz śmieszna rzecz jest taka, że kopiec trudno ładnie formalnie zdefiniować na drzewach.
W sensie najpierw musielibyśmy powiedzieć co to jest poziom w drzewie, co to jest pełny poziom w drzewie a następnie .... w sumie to nawet ja nie wiem jak to ładnie zdefiniować na drzewach :P
Więc zamiast mówić, że kopiec to drzewo które można reprezentować w tablicy, lepiej zdefiniować kopiec jako tablicę na którą możemy patrzeć jako na drzewo.
Wtedy musimy zdefiniować co to jest lewy syn elementu w tablicy, prawy syn oraz ojciec.
Na tej podstawie będzie łatwiej nam zdefiniować własność kopca jako, że dla każdego wierzchołka wartość elementu jest mniejsza od wartości elementów jego dzieci.
Jeśli wolimy zamiast tego powiedzieć że ciąg elementów na ścieżce od liścia do korzenia tworzy ciąg malejący to musimy zdefiniować co to jest liść, korzeń i ścieżka.
Co da się zrobić ale nie wiem czy jest to warte świeczki, gdyż to będzie bodajże jedyne miejsce w których użyjemy tych definicji).}

Kopiec binarny to struktura danych, która reprezentowana jest jako prawie pełne drzewo binarne\footnote{To znaczy wypełniony na wszystkich poziomach (poza, być może, ostatnim).} i na której zachowana jest własność kopca.
Kopiec przechowuje klucze, które tworzą ciąg uporządkowany.
W przypadku kopca typu \emph{min} ścieżka prowadząca od dowolnego liścia do korzenia tworzy ciąg malejący.

Kopce można w prosty sposób reprezentować w tablicy jednowymiarowej \textendash kolejne poziomy drzewa zapisywane są po sobie.

\tizkboxwithcaption{tikz/kopiec-warstwy.tikz}{Reprezentacja kolejnych warstw kopca w tablicy jednowymiarowej.}

Warto zauważyć, że tak reprezentowane drzewo pozwala na łatwy dostęp do powiązanych węzłów.
Synami węzła o indeksie $i$ są węzły $2i+1$ oraz $2i + 2$, natomiast jego ojcem jest $\left\lfloor \frac{i-1}{2} \right\rfloor$.

Kopiec powinien udostępniać trzy podstawowe funkcje: \texttt{zamien\_element}, która podmienia wartość w konkretnym węźle kopca, \texttt{przesun\_w\_gore} oraz \texttt{przesun\_w\_dol}, które zamieniają odpowiednie elementy pilnując przy tym, aby własność kopca została zachowana.

\comment{Da się zrobić aby wierzchołki w ostatniej warstwie się ze sobą nie stykały?
I strzałki są strasznie brzydkie :P
Wydaje mi się, że wystarczy dorobić grot do istniejącej już krawędzi między tymi wierzchołkami i pogrubić ją zamiast robić osobnej strzałki.
Chcemy mieć kolorowe rysunki!
Ja wiem że na czarnobiałej drukarce to i tak wszystko będzie czarnobiałe (duh) ale może chociaż na monitorze komputera będzie jakośładniej :)}
\tizkboxwithcaption{tikz/kopiec-przenoszenie.tikz}{Przykład działania funkcji \texttt{zamien\_element}. a) Oryginalny kopiec. b) Zmiana wartości w wyróżnionym węźle. c) Ponieważ nowa wartość jest większa od wartości swoich dzieci, należy wykonać wywołanie funkcji \texttt{przesn\_w\_dol}. d) Po zmianie własność kopca nie jest zachowana, dlatego należy ponownie wywołać funkcję \texttt{przesn\_w\_dol}. To przywraca kopcowi jego własność.}

\comment{Zamień element jest fajną funkcją, ale funkcje przesun w dol i przesun w gore są ważniejsze.
Ponadto nie chcemy nazywać funkcje w języku polskim.}
\begin{algorithm}[ht]
  \If{k[i] < v}
  {
    k[i] = v\;
	przesun\_w\_dol(k, i)\;
  }
  \Else
  {
    k[i] = v\;
	przesun\_w\_gore(k, i)\;
  }  
  \caption{Implementacja funkcji \texttt{zamien\_element}}
  \label{kopiec-zamien-element}
\end{algorithm}