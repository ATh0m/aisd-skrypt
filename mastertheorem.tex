\section{Twierdzenie o rekursji uniwersalnej}

Twierdzenie o rekursji uniwersalnej pozwala nam ocenić złożoność obliczeniową algorytmu, który wywołuje się rekurencyjnie dla mniejszych danych.

\begin{algorithm}[h]
  \DontPrintSemicolon
  \SetAlgorithmName{Procedura T}{}

  \KwData{ n }
    
  \If{$(n < constant)$}
  {
    return
  }
  Stwórz a podproblemów wielkości n/b w czasie c(n)
  
  
  \For{$i \leftarrow 1$ to $a$}
  {
   	T[n/b]
  }
  Połącz wyniki w czasie d(n)
  \caption{}
  \label{recursion-example}
\end{algorithm}

Złożoność takiego algorytmu możemy zapisać zależnością rekurencyjną
\[T(n) = aT(\frac{n}{b}) + f(n)\]

gdzie 
\(f(n) = c(n) + d(n)\)
\begin{theorem}
 Rekurencja $T(n) = aT(\frac{n}{b}) + f(n)$ może zostać rozwiązana
\begin{itemize}
\item jeżeli $a f(n/b) = c f(n)$ gdzie c < 1, to $T(n) = \Theta(f(n))$
\item jeżeli $a f(n/b) = c f(n)$ gdzie c > 1, to $T(n) = \Theta(n^{log_b a})$
\item jeżeli $a f(n/b) = f(n)$, to $T(n) = \Theta(f(n) log_b n)$
\end{itemize}
 \label{Master}
\end{theorem}
Jeżeli $f(n)$ jest większy o wielokrotność stałej od $a f(b/n)$, wtedy suma jest malejącym ciągiem geometrycznym. Suma każdego ciągu geometrycznego jest wynikiem przemnożenia jej największego elementu przez stałą. W tym przypadku największym elementem jest $f(n)$. \par
Jeżeli $f(n)$ jest mniejszy o wielokrotność stałej od $a f(b/n)$, wtedy suma jest rosnący ciągiem geometrycznym. Suma każdego ciągu geometrycznego jest wynikiem przemnożenia jej największego elementu przez stałą. W tym przypadku, jako, że $f(1) = \Theta(1)$ ostatni element sumy to ilość liści w drzewie rekursji $\Theta(a^{log_b n}) = \Theta(n^{log_b a})$ 
\comment{Tutaj wstawiłbym rysunek drzewka rekursji, ale nie umiem}
\par
Jeżeli $f(n) = a f(b/n)$ to każdy element sumy jest równy $f(n)$.
\subsection{Przykłady wykorzystania twierdzenia}
\begin{itemize}
\item Mergesort: T(n) = 2 * T(n/2) + n log(n)
\par $a * f(n/b) = 2 * n/2 = n = f(n)$, więc $T(n) = \Theta(n log n)$
\item Mergesort z dziwnym mergem(przykład z próbnego egzaminu): T(n) = 2 * T(n/2) + n log(n)
\par $a * f(n/b) = 2 * n/2 * log(n/2) = n * log(n/2) = n * (log(n) - 1)$, nie możemy zastosować tutaj prostej równości, jednak zauważmy, iż przy odpowiednio dużym n wartość ta jest bliska $f(n)$; tj. $\lim_{x\to\infty} \frac{n log(n)}{n (log(n) - 1)} = 1$, więc $T(n) = n * log(n) * log(n) = n * log^{2}(n)$
\item Mergesort z mergem w czasie stałym(przykład z próbnego egzaminu): T(n) = 2 * T(n/2) + 1
\par  $a * f(n/b) = 2$, więc nasze twierdzenie nie będzie miało zastosowania. Można jednak łatwo zauważyć, że $T(n) = \Theta (n)$ i udowodnić to indukcyjnie.
\item Algorytm Karatsuby: T(n) = 3 * T(n/2) + n
$a * f(n/b) = \frac{3}{2} * n$; $\frac{3}{2} > 1$, więc $T(n)=n^{log_2(3)}$.
\end{itemize}