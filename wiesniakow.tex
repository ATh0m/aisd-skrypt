\section{Algorytm rosyjskich wieśniaków}

\begin{algorithm}[h]
  \DontPrintSemicolon
  
  \KwData{ $a$, $b$ - liczby naturalne }
  
  \KwResult{ $wynik = a \cdot b$ }
  
  $a' \leftarrow a$\;
  $b' \leftarrow b$\;
  $wynik \leftarrow 0$\;
  \While{\upshape $a' > 0$}
  {
    \If{\upshape $a' \textsf{ mod } 2 = 1$}
    {
      $wynik \leftarrow wynik + b'$\;
    }
    $a' \leftarrow a' \textsf{ div } 2$\;
    $b' \leftarrow b' \cdot 2$\;
  }
  
  \caption{Algorytm rosyjskich wieśniaków}
  \label{alg-wiesniakow}
\end{algorithm}

Niech $a'_i$ (kolejno: $b'_i$, $wynik_i$) będzie wartością $a'$ ($b'$, $wynik$) w $i-tej$ iteracji pętli \texttt{while}. Udowodnimy następujący niezmiennik: $a'_i \cdot b'_i + wynik_i = a \cdot b$. Załóżmy, że niezmiennik zachodzi w $i-tej$ iteracji i sprawdźmy co dzieje się w $i+1$ iteracji. Rozważmy dwa przypadki.


\begin{itemize}
    \item $a'_i$ parzyste. Instrukcja \texttt{if} się nie wykona, w $i+1$ iteracji $wynik_i$ pozostanie niezmieniony, $a'_i$ zmniejszy się o połowę, a $b'_i$ zwiększy dwukrotnie. 

$a'_{i+1} \cdot b'_{i+1} + wynik_{i+1} = \frac{a'_i}{2} \cdot 2 b'_i + wynik_i = a_i \cdot b_i$


\item $a'_i$ nieparzyste:

$wynik_{i+1} \leftarrow wynik_i + b'_i$;  
$a'_{i+1} \leftarrow a'_i \textsf{ div } 2 = \frac{a'_i-1}{2}$;  
$b'_{i+1} \leftarrow b'_i \cdot 2$

Ostatecznie otrzymujemy:

$a'_{i+1} \cdot b'_{i+1} + wynik_{i+1} = \frac{a'_i-1}{2} \cdot 2 b'_i + wynik_i +b'_i = a'_i \cdot wynik_i + b'_i= a \cdot b$

\end{itemize}

Teraz wystarczy zauważyć, że tuż po wyjściu z pętli \texttt{while} wartość zmiennej $a'$ wynosi $0$. Podstawiając do niezmiennika okazuje się, że faktycznie algorytm rosyjskich wieśniaków liczy $a \cdot b$.

\paragraph{Złożoność}

Z każdą iteracją połowimy $a'$. Biorąc pod uwagę kryterium jednorodne pozostałe instrukcje w pętli nic nie kosztują. Stąd złożoność to $O(\log a)$.

W kryterium logarytmicznym musimy uwzględnić czas dominującej instrukcji: dodawania  $wynik \leftarrow wynik + b'$. W najgorszym przypadku zajmuje ono $O(\log ab)$. Zatem złożoność to $O(\log a \cdot \log ab)$.