\section{Algorytm rosyjskich wieśniaków}



Algorytm rosyjskich wieśniaków jest przypisywany sposobowi mnożenia liczb używanemu w XIX-wiecznej Rosji.
Aktualnie jest on stosowany w niektórych układach mnożących.

W celu obliczenia $a \cdot b$ tworzymy tabelkę i liczby $a$ i $b$ zapisujemy w pierwszym jej wierszu.
Kolumnę $a$ wypełniamy następująco: w $i+1$ wierszu wpisujemy wartość z wiersza $i$ podzieloną całkowicie przez 2.
W kolumnie $b$ kolejne wiersze tworzą ciąg geometryczny o ilorazie równym 2.
Wypełnianie tabelki kończymy wtedy, gdy w kolumnie $a$ otrzymamy wartość 1.
Na koniec sumujemy wartości w kolumine pod $b$ z tych wierszy dla których wartości w kolumine pod $a$ są nieparzyste.
Uzyskany wynik to $a \cdot b$.

W poniższym przykładzie obliczymy $42 \cdot 17$.

\begin{center}
\begin{tabular}{ c|c } 

 a & b \\ 
 \hline
 42 & 17 \\ 
 \textbf{21} & $17 \cdot 2 = 34$ \\
 10 & $17 \cdot 2^2 = 68$ \\
 \textbf{5}  & $17 \cdot 2^3 = 136$ \\
 2  & $17 \cdot 2^4 = 272$ \\
 \textbf{1}  & $17 \cdot 2^5 = 544$ \\

\end{tabular}
\end{center}
\comment{Jak usuniecie enter po tabelce, to to zdanie nie dostanie w pdfie tabulatora.
Tabulator powinien się znajdować przed akapitami.
To zdanie jest częścią poprzedniego akapitu.}
$a$ jest nieparzyste w wierszu 2, 4 oraz 6. Zatem będziemy sumować wartości $b$ z wiersza 2, 4 i 6.

\begin{equation*} 
\begin{split}
a \cdot b &= 17\cdot2 + 17\cdot2^3 + 17\cdot2^5 \\
&= 34 + 136 + 544 \\
&= 714
\end{split}
\end{equation*}
Faktycznie, otrzymaliśmy wynik poprawny. Spójrzmy raz jeszcze na na tę sumę:

\begin{equation*} 
\begin{split}
a \cdot b &= 17\cdot2 + 17\cdot2^3 + 17\cdot2^5 \\
&= 17 \cdot ( 2^5 + 2^3 + 2) \\
&= 17 \cdot ( 1 \cdot 2^5 + 0 \cdot 2^4 + 1 \cdot 2^3 + 0 \cdot 2^2 + 1 \cdot 2^1 + 0 \cdot 2^0 ) \\
&=17_{10} \cdot 101010_2 \\
&=17 \cdot 42 = 714
\end{split}
\end{equation*}

Przypomnij sobie algorytm zamiany liczby w systemie dziesiętnym na system binarny.
Okazuje się, że algorytm rosyjskich wieśniaków "po cichu" wylicza tę reprezentację $a$.

W kolejnych paragrafach podamy algorytm i w celu jego udowodnienia sformułujmy \textit{niezmiennik} oraz wykażemy jego prawdziwośc.

\begin{algorithm}[h]
  \DontPrintSemicolon
  \SetAlgorithmName{Algorytm}{}
  
  \KwData{ $a$, $b$ - liczby naturalne }
  
  \KwResult{ $wynik = a \cdot b$ }
  
  $a' \leftarrow a$\;
  $b' \leftarrow b$\;
  $wynik \leftarrow 0$\;
  \While{\upshape $a' > 0$}
  {
    \If{\upshape $a' \textsf{ mod } 2 = 1$}
    {
      $wynik \leftarrow wynik + b'$\;
    }
    $a' \leftarrow a' \textsf{ div } 2$\;
    $b' \leftarrow b' \cdot 2$\;
  }
  
  \caption{Algorytm rosyjskich wieśniaków}
  \label{alg-wiesniakow}
\end{algorithm}

\begin{theorem}
Niech $a'_i$ (kolejno: $b'_i$, $wynik_i$) będzie wartością zmiennej \texttt{a'} (\texttt{b'}, \texttt{wynik}) w $i-tej$ iteracji pętli \texttt{while}. Zachodzi następujący niezmiennik pętli:
\[
a'_i \cdot b'_i + wynik_i = a \cdot b \enspace.
\]
\end{theorem}

\begin{lemma}
Przed wejściem do pętli \texttt{while} niezmiennik jest prawdziwy.
\end{lemma}
\begin{proof}
Skoro przed przed wejściem do pętli mamy: $a'_0 = a$, $b'_0 = b$ oraz $wynik_0 = 0$, to oczywiście: $a'_0 \cdot b'_0 + wynik_0 = a \cdot b + 0 = a \cdot b$.
\end{proof}

\begin{lemma}
Po $i-tym$ obrocie pętli niezmiennik jest spełniony.
\end{lemma}
\begin{proof}
Załóżmy, że niezmiennik zachodzi w $i-tej$ iteracji i sprawdźmy co dzieje się w $i+1$ iteracji.
\comment{Nie możesz czegoś takiego założyć :)}
Rozważmy dwa przypadki.


\begin{itemize}
    \item $a'_i$ parzyste. Instrukcja \texttt{if} się nie wykona, w $i+1$ iteracji $wynik_i$ pozostanie niezmieniony, $a'_i$ zmniejszy się o połowę, a $b'_i$ zwiększy dwukrotnie. 
    \[
      wynik_{i+1} = wynik_i
    \]
    \[
      a'_{i+1} = a'_i \textsf{ div } 2 = \frac{a'_i}{2}
    \]
    \[
      b'_{i+1} = b'_i \cdot 2
    \]
    W tym przypadku otrzymujemy:
    \[
      a'_{i+1} \cdot b'_{i+1} + wynik_{i+1} = \frac{a'_i}{2} \cdot 2 b'_i + wynik_i = a'_i \cdot b'_i + wynik_i = a \cdot b
    \]

    \item $a'_i$ nieparzyste:
    \[
      wynik_{i+1} = wynik_i + b'_i
    \]
    \[
      a'_{i+1} = a'_i \textsf{ div } 2 = \frac{a'_i-1}{2}
    \]
    \[
      b'_{i+1} = b'_i \cdot 2
    \]
    Ostatecznie otrzymujemy:
    \[
      a'_{i+1} \cdot b'_{i+1} + wynik_{i+1} = \frac{a'_i-1}{2} \cdot 2 b'_i + wynik_i +b'_i = a'_i \cdot wynik_i + b'_i= a \cdot b
    \]

\end{itemize}

\end{proof}

\begin{lemma}
Po zakończeniu algorytmu $wynik = a \cdot b$

\end{lemma}
\begin{proof}
Wystarczy zauważyć, że tuż po wyjściu z pętli \texttt{while} wartość zmiennej $a'$ wynosi $0$.
Podstawiając do niezmiennika okazuje się, że faktycznie algorytm rosyjskich wieśniaków liczy $a \cdot b$.
\end{proof}

\begin{lemma}
Algorytm sie kończy.
\end{lemma}
\begin{proof}
Skoro $a_i \in \mathbb{N} $ oraz $\mathbb{N}$ jest dobrze uporządkowany, to połowiąc $a_i$ po pewnej liczbie iteracji otrzymamy 0.
\end{proof}

Z powyższych lematów wynika, że niemiennik spełniony jest zarówno przed, w trakcie jak i po zakończeniu algorytmu. 
Algorytm rosyjskich wieśniaków jest poprawny.

\paragraph{Złożoność}

Z każdą iteracją połowimy $a'$. 
Biorąc pod uwagę kryterium jednorodne pozostałe instrukcje w pętli nic nie kosztują. 
Stąd złożoność to $O(\log a)$.

W kryterium logarytmicznym musimy uwzględnić czas dominującej instrukcji: dodawania  $wynik \leftarrow wynik + b'$. 
W najgorszym przypadku zajmuje ono $O(\log ab)$. Zatem złożoność to $O(\log a \cdot \log ab)$.
